%%% AUTHOR: BAHATTIN

\subsection{Interaction Viewpoint}
\paragraph{}
\normalsize
This section of the Software Design Document explains interactions between several objects of the project. Below, interactions happening on operations like loading data, applying processors, creating pipelines can be seen. A simple written explanation is given with diagrams.\\

\skipsubsection

\subsubsection{Loading data}
\paragraph{}
\normalsize
When LoadData process is initiated with a file name, first thing it does is to open given file. After that, it gives control to ProcessorManager via a CanReadFormat call. ProcessorManager forwards this call to registered "input" type Processors. First available input Processor is generated with the given filename and added to the Pipeline. This Pipeline object is then returned.\\

\begin{tikzpicture} 
\begin{umlseqdiag} 
\umlobject{LoadFile} 
\umlcreatecall[class=Pipeline]{LoadFile}{pl}
\umlobject[class=ProcessorManager]{PM} 
\umlmulti[class=Processor]{pr} 
\begin{umlcallself}[op={open given file}]{LoadFile}
\end{umlcallself}
\begin{umlcall}[op={CanReadFormat(filename,handle)}, return=bool]{LoadFile}{PM} 
	\begin{umlcall}[op={GetType}, return={[input,process,sink]}]{PM}{pr} 
	\end{umlcall} 
	
	\begin{umlcall}[op={CanReadFormat}, return=bool]{PM}{pr} 
	\end{umlcall} 
\end{umlcall} 
\begin{umlcall}[op={GetReader(filename,handle)}, return=pr]{LoadFile}{PM}
	\begin{umlcall}[op={GetType}, return={[input,process,sink]}]{PM}{pr} 
	\end{umlcall} 
	
	\begin{umlcall}[op={CanReadFormat}, return=bool]{PM}{pr} 
	\end{umlcall} 
\end{umlcall}
\begin{umlcall}[op={AddProcessor(pr)}, return=bool]{LoadFile}{pl}
\end{umlcall} 
\end{umlseqdiag} 
\end{tikzpicture}

\subsubsection{Applying processors}
\paragraph{}
\normalsize
Processors are bound to Pipeline objects. However, they can be called without being bound. This flow explains how a Pipeline applies Processors. Pipeline object will generate an Packet object and follow Processor chain. A Processor is free to do whatever it wants on a Packet.\\

\begin{tikzpicture} 
\begin{umlseqdiag} 
\umlobject[class=Pipeline]{pl} 
\umlmulti[class=Processor]{pl->processors} 
\umlobject[class=Packet]{input} 
\begin{umlcall}[op={Process(input)}, return=Packet]{pl}{pl->processors}
	\begin{umlcall}[op={GetData}, return={double[][]}]{pl->processors}{input} 
	\end{umlcall} 
	
	\begin{umlcall}[op={GetExtra}, return=Box]{pl->processors}{input} 
	\end{umlcall} 
\end{umlcall} 
\end{umlseqdiag}
\end{tikzpicture}

\subsubsection{Creating pipelines}
\paragraph{}
\normalsize
This diagram assumes named Processor creation from array. ProcessorManager finds wanted Processor and forwards call to it.\\

\begin{tikzpicture} 
\begin{umlseqdiag} 
\umlobject{main} 
\umlcreatecall[class=Pipeline]{main}{pl}
\umlobject[class=ProcessorManager]{PM} 
\umlmulti[class=Processor]{pr} 
\begin{umlcall}[op={FromArray(array)}, return=Processor]{main}{PM} 
	\begin{umlcallself}[op={find given processor}]{PM}
	\end{umlcallself}
	
	\begin{umlcall}[op={FromArray(array[1:])}, return={pr:Processor}]{PM}{pr} 
	\end{umlcall} 
\end{umlcall} 
\begin{umlcall}[op={AddProcessor(pr)}, return=bool]{main}{pl}
\end{umlcall} 
\end{umlseqdiag} 
\end{tikzpicture}


\skipsubsection
