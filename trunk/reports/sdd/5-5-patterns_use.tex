%%% AUTHOR: BARIS

\subsection{Patterns Use Viewpoint}
\paragraph{}
\normalsize
    In this part of the design document, it will be explained that how subsystem will be connected and in which order their functions will be called. The order of the function can be seen in Collabration Diagram. First of all, duties of functions in the diagram will be explained.\\
    
    \begin{tabular}{|p{4cm}|p{12cm}|}
     	\hline
        	\textbf{Function Name} & \textbf{Function Duty} \\
        \hline
            loadFile() & This function is called by user to load the data taken by fMRI machine. This function starts the Input subsystem.\\
        \hline
            downsampleData() & This function is called by Input subsytem to connect the Filtering subsystem to downsample the given data to minify it.\\
        \hline
            quantizeData() & This function is called by the Filtering subsystem after execution of downsampleData() to reduce the file size and ease the handle of data. This function does quantization on input.\\
        \hline
            showBrain() & This function is called by Filtering subsytem to connect the Visualization subsystem after execution of quantizeData() function. This function shows the processed data as a 3D image. This function uses built-in Unity3D functions and OpenGL libraries and function implemented by the Simple Labs.\\
        \hline
        	changeDisplay() & This function is called if the user adjusts transparency, colors, rotation, zoom or layer depth or changes the display by clicking on "Show Side-by-Side" or "Four Regions" button. This function cannot be called before the showBrain() function.\\
        \hline
    \end{tabular}
    
    \begin{tikzpicture} 
    
    \umlactor[x=5, y=0, scale=2]{User}
    
    \begin{umlstate}[x=0, y=-7, name=Input]{Input} 
	\end{umlstate} 
    
    \begin{umlstate}[x=0, y=-14]{Filtering} 
    \end{umlstate} 
    
    \begin{umlstate}[x=10, y=-10.5]{Visualization} 
    \end{umlstate} 
    
    \umltrans[recursive=-20|-60|1.5cm, recursive direction=right to bottom, arg={6:updateDisplay()}, pos=1.5]{Visualization}{Visualization}
    \umltrans[arg={5:changeDisplay()}]{User}{Visualization}
    \umltrans[arg={4:showBrain()}]{Filtering}{Visualization}
    \umltrans[recursive=-10|10|4cm, arg={3:quantizeData()}, pos=1.5, recursive direction=right to right]{Filtering}{Filtering}  
    \umltrans[arg={2:downsampleData()}]{Input}{Filtering}
    \umltrans[arg={1:loadFile()}]{User}{Input}
    
    \end{tikzpicture}\\
    
    \textbf{Order of Function}\\
 
		The function is called with respect to numbers stated in diagram.\\
    
    	Firstly, first function is called and input data is loaded to the system. Secondly, second and third functions are called and game is started. After that, fourth function is called to show the 3D image created with the processed data. Lastly, fifth and sixth functions are called repeatedly when there is a user interaction until the program is closed.\\

    \newpage
\skipsubsection

