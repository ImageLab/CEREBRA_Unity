%%% AUTHOR: ATAKAN

\newpage

\section{Design Description Information Content}

\subsection{Introduction}
\paragraph{}
\normalsize
    This is an SDD document for Cognitive State Representation and Visalization of Human Brain project. Thorought the document, detailed design cases are explained and depicted using UML diagrams. 
\skipsubsection
    

\subsection{SDD Identification}
\paragraph{}
\normalsize
    This Software Design Document is written on the request of Ceng 491 instructors to be able to guide the development process of Cognitive State Representation and Visalization of Human Brain. It is written by Simple Labs team collaboratively. Below table presents the authorships of sections. 
    
    \begin{table}[h]
    \begin{tabular}{|l|l|}
    \hline
    \textbf{Section} & \textbf{Author} \\
    \hline
    1.* & Barış Nasır                                                     \\ \hline
    2.* & Barış Nasır                                                     \\ \hline
    3.* & Özlem Ceren Şahin                                               \\ \hline
    4.* & Atakan Kaya                                                     \\ \hline
    5.1 & Bahattin Tozyılmaz                                              \\ \hline
    5.2 & Atakan Kaya                                                     \\ \hline
    5.3 & Bahattin Tozyılmaz, Barış Nasır, Özlem Ceren Şahin, Atakan Kaya \\ \hline
    5.4 & Özlem Ceren Şahin                                               \\ \hline
    5.5 & Barış Nasır                                                     \\ \hline
    5.6 & Atakan Kaya                                                     \\ \hline
    5.7 & Bahattin Tozyılmaz                                              \\ \hline
    5.8 & Bahattin Tozyılmaz, Barış Nasır                                 \\ \hline
    \end{tabular}
\end{table}
    
    The date of issue of the initial version of this document is \today . 
\skipsubsection
    

\subsection{Design Stakeholders and Their Concerns}
\paragraph{}
\normalsize
The design stakeholders for our project are Prof. Dr. Fatoş Yarman Vural and her research group. Our project is shaped by their research and requirements.

The major concerns of design stakeholders can be listed as:
\begin{itemize}
	\item They want to observe brain using this tool. 
    \item They want the edge size to be reduced and a smooth 3D image to be rendered.
	\item They want the project to be completed in time. 
    \item They want to be kept informed about the process.
\end{itemize}

\skipsubsection
    

\subsection{Design Views}
\paragraph{}
\normalsize
Our project has emerged from a need of an efficient, simple and smooth drawing of brain data. To achieve this, filtering techniques are required. And since the filtering techniques are under research and development phase by the Image Processing Laboratory of METU, an object-oriented and an easily extendable approach is preferred. To point this, a contextual view that determines the services required, a logical view that draws the relations between basic entities, a dependency view and a patterns use view that defines the relation between subsystems, an interface view that gives insight about how the end product will be, an interaction view that depicts the flow of information and an algorithm view that focuses on the algorithms used is reqired. 
\skipsubsection
 
 
\subsection{Design Viewpoints}
\paragraph{}
\normalsize
In this document, the contextual viewpoint focuses on the services by use case diagrams to define the usage of features by the actors. Then, a logical viewpoint defines the classes and the relationships between them. In the dependency viewpoint, the relationships of interconnections and access among entities are specified. Later, patterns use viewpoint depicts how subsystems of the project are connected. In the interface viewpoint the relations of the UI modules and a mockup visualization is provided. Then, interaction viewpoint explains the interactions between several objects of the project. Finally, algorithm viewpoint defines the required algorithms throughout the project.
\skipsubsection


\subsection{Design Rationale}
\paragraph{}
\normalsize
	In this document, design features are chosen to improve reusability and provide extensibility. This is vital since the related projects at Image Processing Laboratory are under development, the reqirements for them can change.
\skipsubsection


\subsection{Design Languages}
\paragraph{}
\normalsize
	Throughout the document, UML use case diagrams, UML component diagrams, UML class diagrams, UML sequence diagrams and ER diagrams are used.
\skipsubsection
