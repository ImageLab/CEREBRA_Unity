%%% AUTHOR: BARIS, ATAKAN, OZLEM, BAHATTIN

\subsection{Logical Viewpoint}
\paragraph{}
\normalsize
This section of the Software Design Document focuses on logical foundations of the project. The logical foundations of the project are the classes and relations between them. In this section, those classes, their methods and interactions will be explained in detail.\\

Extensibility is a must for the project. Since this project will be used in a highly active research area, it is essential that novel ideas be implemented easily. Project team aims to achieve this with a highly algorithm and data agnostic approach.\\
\skipsubsection


\subsubsection{Box Class}
\paragraph{}
\normalsize
This class is used to pass arbitrary data between Processors. Box class is mainly responsible for data agnosticity. It achieves this by encapsulating data and its type together. This class depends heavily on C++'s RTTI and templating abilities. This class will box a value or POD structure and will unbox it only when its original type is known.\\

\begin{center}
	\begin{tikzpicture} 
	\umlclass{Box}{ 
		-- data : void* \\ 
		-- type : string
		}{
		+ Box() : constructor \\  
		+ Box<T>(d : T) : constructor \\ 
		+ isEmpty() : boolean \\ 
		+ operator=<T>(d : T) : T \\ 
		+ UnBox<T>(d: T) : boolean
	} 
	\end{tikzpicture}
	\\[0.8cm]
	\begin{tabular}{|p{2.2cm}|p{2.2cm}|p{2cm}|p{7.6cm}|}
		\hline 
			Name & Returns & Visibility & Description \\
		\hline 
			data & void* & Private & This member holds a pointer to copied POD data \\
		\hline 
			type & string & Private & This member is the copy of the type string returned by RTTI \\
		\hline
			Box() & <<constructor>> & Public & Dummy constructor. Instances initialized this way are empty until they are assigned a value \\
		\hline
			Box<T>(T) & <<constructor>> & Public & Copy constructor \\
		\hline
			isEmpty() & boolean & Public & Checks whether the Box is assigned \\
		\hline
			operator=(T) & T & Public & Sets Box content to right hand side (by copying data) \\
		\hline
			UnBox<T>(T) & boolean & Public & Tries to unbox the value. If the provided type is not correct, argument is left untouched and false is returned. Else, argument is written with data and true is returned.\\
		\hline
	\end{tabular}
\end{center}

\skipsubsection

\subsubsection{Packet Class}
\paragraph{}
\normalsize
This class is used as an immediate data format between two Processors. Packet class encapsulates all data needed by Processors: voxel coordinates, edge values, etc... It also offers a way to pass named extra data between Processors.\\

\begin{center}
	\begin{tikzpicture} 
	\umlclass{Packet}{ 
		-- voxelCoordinates : double[][3] \\ 
		-- edges : double[][]
		-- extras : map<string,Box>
		}{
		+ Packet() : constructor \\  
		+ Packet(d : Packet) : constructor \\ 
		+ SetExtra<T>(name: string, data: T) : T \\
		+ GetExtra<T>(name: string, data: T) : bool \\
		+ operator[](name: string) : Box \\
		+ GetEdges(n: int) : double[][] \\
		+ SetEdges(data: double[][]) : double[][] \\
		+ GetCoords(n: int) : double[][3] \\
		+ SetCoords(data: double[][3]) : double[][3] \\
	} 
	\end{tikzpicture}
	\\[0.8cm]
	\begin{tabular}{|p{4.5cm}|p{3cm}|p{1.5cm}|p{5cm}|}
		\hline 
			Name & Returns & Visibility & Description \\
		\hline 
			voxelCoordinates & double[][3] & Private & This member holds coordinates of voxels \\
		\hline 
			edges & double[][] & Private & This member holds edge matrix \\
		\hline
			extras & map<string,Box> & Private & Named collection of extra datas \\
		\hline
			Packet() & <<constructor>> & Public & Dummy constructor \\
		\hline
			Packet(d: Packet) & <<constructor>> & Public & Copy constructor \\
		\hline
			SetExtra<T>(name: string, data: T) & T & Public & Sets an extra with the given name and content \\
		\hline
			GetExtra<T>(name: string, data: T) & bool & Public & Gets the extra with the given name or returns false \\
		\hline
			operator[](name: string) & Box & Public & Shorthand for getting and setting extras \\
		\hline
			GetEdges(n: int) & double[][] & Public & Gets the edge matrix \\
		\hline
			SetEdges(data: double[][]) & double[][] & Public & Sets the edge matrix \\
		\hline
			GetCoords(n: int) & double[][3] & Public & Gets voxel coordinates \\
		\hline
			SetCoords(data: double[][3]) & double[][3] & Public & Sets voxel coordinates \\
		\hline
	\end{tabular}
\end{center}

\skipsubsection

\subsubsection{Processor Interface}
\paragraph{}
\normalsize
This interface defines outlines of Processors and how Processors should be implemented. \\

\begin{center}
	\begin{tikzpicture} 
	\umlclass[type=interface]{Processor}{ 
		}{
		+ Processor() : constructor \\  
		+ Processor(arg: string[]) : constructor \\ 
		+ FromArray(arg: string[]) : void \\ 
		+ Process(input : Packet) : Packet \\ 
		+ GetType() : string \\
		+ GetName() : string \\
		+ GetInfo() : string \\
	} 
	\end{tikzpicture}
	\\[0.8cm]
	\begin{tabular}{|p{4cm}|p{2.2cm}|p{1.8cm}|p{6cm}|}
		\hline 
			Name & Returns & Visibility & Description \\
		\hline 
			Processor() & <<constructor>> & Public & Dummy constructor \\
		\hline 
			Processor(arg: string[]) & <<constructor>> & Public & This should be the constructor called from other places \\
		\hline
			FromArray(arg: string[]) & void & Public & Sets properties of the Processor \\
		\hline
			Process(input: Packet) & Packet & Public & The real job is done here \\
		\hline
			GetType() & string & Public & Returns "sink", "process" or "input" \\
		\hline
			GetName() & string & Public & Returns internal name for the Processor \\
		\hline
			GetInfo() & string & Public & Returns a simple documentation\\
		\hline
	\end{tabular}
\end{center}

\skipsubsection

\subsubsection{Pipeline Class}
\paragraph{}
\normalsize
This class is responsible for chaining Processor operations. A pipeline is an object that the use can save and load from a file. Thus, it also gives the user an option for presets. When a Processor is added to the Pipeline, Pipeline object checks whether it is the first Processor to be added, and if it is, is it an input type Processor.\\

\begin{center}
	\begin{tikzpicture} 
	\umlclass{Pipeline}{
		-- processors: Processor[]
		}{
		+ Pipeline() : constructor \\  
		+ Pipeline(rhs: Pipeline) : constructor \\ 
		+ FromArray(arg: string[]) : void \\ 
		+ ToArray() : string[] \\ 
		+ AddProcessor(pr: Processor) : bool \\ 
		+ Run() : Packet \\
	} 
	\end{tikzpicture}
	\\[0.8cm]
	\begin{tabular}{|p{4cm}|p{2.2cm}|p{1.8cm}|p{6cm}|}
		\hline 
			Name & Returns & Visibility & Description \\
		\hline 
			processors & Processor[] & Private & This is the list of Processors this Pipeline consists of \\
		\hline 
			Pipeline() & <<constructor>> & Public & Dummy constructor \\
		\hline 
			Pipeline(rhs: Pipeline) & <<constructor>> & Public & Copy constructor \\
		\hline
			FromArray(arg: string[]) & void & Public & Constructs "processors" with given information \\
		\hline
			ToArray() & string[] & Public & Saves "processors" list so that it can be loaded with FromArray\\
		\hline
			AddProcessor(pr: Processor) & bool & Public & Adds a Processor to the list. First Processor on the list should be input type \\
		\hline
			Run() & Packet & Public & Runs generated Processor sequence, returns output of last Processor\\
		\hline
	\end{tabular}
\end{center}

\skipsubsection

\subsubsection{ProcessorManager Class}
\paragraph{}
\normalsize
This class is responsible for managing Processor selection and generation. This is a static class and it's members are all static. Each Processor must register itself with the ProcessorManager. C++ doesn't allow static constructors, which would be used when registering. This is a problem the team is working on.\\

\begin{center}
	\begin{tikzpicture} 
	\umlclass[type=static]{ProcessorManager}{
		-- processors: Processor[]
		}{
		+ ProcessorManager() : constructor \\  
		+ Register(p :Processor) : void \\ 
		+ FromArray(arg: string[]) : void \\ 
		+ GetReader(fn: string, h: Handle) : Processor
	} 
	\end{tikzpicture}
	\\[0.8cm]
	\begin{tabular}{|p{4cm}|p{2.2cm}|p{1.8cm}|p{6cm}|}
		\hline 
			Name & Returns & Visibility & Description \\
		\hline 
			processors & Processor[] & Private & This is the list of registered Processors\\
		\hline 
			ProcessManager() & <<constructor>> & Public & Static constructor \\
		\hline 
			Register(p: Processor) & void & Public & Registers p with ProcessManager \\
		\hline
			FromArray(arg: string[]) & void & Public & Constructs a Processor with given information \\
		\hline
			GetReader(fn: string, h: Handle) & Processor & Public & Finds the input Processor that can read given source\\
		\hline
	\end{tabular}
\end{center}

\skipsubsection