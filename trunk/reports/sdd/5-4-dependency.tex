%%% AUTHOR: OZLEM

\subsection{Dependency Viewpoint}
\paragraph{}
\normalsize
	In this section of the design document, the relationships of interconnections and access among entities are specified. These relationships include information sharing, order of execution and parameterization of interfaces.
        
        ER diagram below shows the entities and their relationships. They are also explained in the subsections of this section.
       
\usetikzlibrary{positioning}
\usetikzlibrary{shadows}

\tikzstyle{every entity} = [top color=white, bottom color=blue!30, 
                            draw=blue!50!black!100, drop shadow]
\tikzstyle{every relationship} = [top color=white, bottom color=red!20, 
                                  draw=red!50!black!100, drop shadow]
\tikzstyle{every attribute} = [top color=white, bottom color=yellow!20, 
                               draw=yellow, node distance=1cm, drop shadow]
                               
\scalebox{.60}{
\begin{tikzpicture}

	\node[entity] (input) {Input} [<-];
    
	\node[relationship] (require1) [below=of input] {Requires} edge (input);
    
    \node[entity] (loadFile) [below=of require1] {LoadFile} edge (require1);
    
    \node[relationship] (produce1) [left=of loadFile] {Produces} edge (loadFile);
    \node[relationship] (produce2) [right=of loadFile] {Produces} edge (loadFile);
    \node[relationship] (produce3) [below=of loadFile] {Produces} edge (loadFile);
    
    \node[entity] (procMang) [below=of produce1] {ProcessorManager} edge (produce1);
    \node[attribute] (processors1) [left=of procMang] {Processors} edge (procMang);
    \node[relationship] (provide1) [below=of procMang] {Provides} edge (procMang);
    
    \node[entity] (processor) [below=of provide1] {Processor} edge (provide1);
    
    
    \node[entity] (packet) [right=of produce2] {Packet} edge (produce2);
    \node[relationship] (uses) [right=of packet] {Uses} edge (packet);
    
    
    
     
    
    \node[entity] (box) [right=of uses] {Box} edge (uses);
    
    \node[attribute] (data) [above right=of box] {Data} edge (box);
    \node[attribute] (type) [below right=of box] {Type} edge (box);
    
    
    \node[entity] (pipeLine) [below=of produce3] {PipeLine} edge (produce3);
   
    \node[relationship] (provides2) [right=of pipeLine] {Requires and Provides} edge (pipeLine);
    
    
    \node[relationship] (require2) [right=of processor] {Requires} edge (processor);
    \draw[link] (require2) -| (pipeLine) ;
    \node[relationship] (require3) [right=of processor] {Requires} edge (processor);
    
    \draw[link] (provides2) -| (packet);
 
   
   
    \node[relationship] (provides3) [right=of pipeLine] {Requires and Provides} edge (pipeLine);
    
    \node[attribute] (processors2) [below left=0.5cm of provides3] {Processors} edge (pipeLine);
    
      \node[relationship] (requires3) [above=of packet] {Requires} edge (packet);

    \node[entity] (visual) [above=of requires3] {Visualization} edge (requires3);



\end{tikzpicture}
}
        


\paragraph{}
\normalsize

Dependency viewpoint provides an overall picture of the system entities and their relationships in order to assess the impact of requirements and design changes. This section helps maintainers in two ways: System failures or resource bottlenecks can be resolved by identifying the entities which causes them and development plan can be prepared by identifying which entities are needed by other entities and which should be developed first. 

\paragraph{}
\normalsize
There are seven design entities which are Input, Box, Packet, Processor, ProcessorManager, LoadFile and Pipeline.
    
There are four design relationships, namely  uses,requires, provides, produces.
\begin{itemize}
\item	uses : Packet uses Box to store some of its attributes.
\item    requires: In the main loadFile requires input from the user, Pipeline requires Packet of preprocessed input data and one or more Processors to process Packet.
\item    provides: Pipeline provides Packet at the end of its process and ProcessorManager generates processors and provides them for further use.
\item  	produces: loadFile produces Packet, ProcessorManager and Pipeline due to the input, which includes user choices which effects attributes of these entities.
\end{itemize}

\paragraph{}
\normalsize
Short descriptions of attributes are given below but detailed information about attributes can be found at section 5.3 Logical Viewpoint.
\begin{itemize}
\item	Box
 
  \begin{itemize}
  	\item	data : pointer to copied POD data.
    \item	type : copy of the type string returned by RTTI.
  \end{itemize}

\item    Packet
 \begin{itemize}
  	\item	voxelCoordinates: Coordinates of the brain voxels.
    \item	edges : Edge matrix.
  \end{itemize}

\item    ProcessorManager
\begin{itemize}
  	\item	processors : list of processors.
  \end{itemize}	
\item  	PipeLine
\begin{itemize}
  	\item	processors : list of processors.
  \end{itemize}
\end{itemize}


\skipsubsection


