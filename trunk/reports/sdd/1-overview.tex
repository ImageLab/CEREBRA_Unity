%%% AUTHOR: BARIS

\newpage

\section{Overview}
\subsection{Scope}
\paragraph{}
\normalsize
The software to be produced is “Cognitive State Representation and Visualization of Human Brain”. In this project it is aimed to connect different simulation, computer graphics and image processing technologies. At the end of the project, a software enhancing the graph structure (by simplifying it with down sampling and quantization) and Unity3D game engine will be implemented. To visualize our work and make the software more attractive, the project is designed to include some extra features like zoom in/out, rotation, showing brain parts seperately. \\
    
    Initially, the project is planned to be fully implemented with OpenGL. However, after meeting with the customer and our supervisor, we decided to use a game engine to implement visual part of the project. After doing some researches we decided to use Unity3D game engine.\\
    
    Unity3D game engine is selected for this project because it includes occlusion culling feature that renders only what can be seen, level of detail support and build size stripping. Also, it supports DirectX 11, shader model 5.0 and OpenGL.\\
\skipsubsection
    
    
\subsection{Purpose}
\paragraph{}
\normalsize
This document describes how “Cognitive State Representation and Visualization of Human Brain” will be structured to satisfy the requirements identified in the Software Requirements Specification document prepared by Simple Labs. Team in their senior software project. It includes modifications over initial design document.\\
    
	Requirements Specification document determines software, hardware, functional and nonfunctional requirements decided to be satisfied and gives a general idea how the system will work. This document covers the details and different aspects of the project in a comprehensive way and conceptualizes the overall product that will be formed accurately.\\
    
    In the design process, it is intended to design an effective and modular product that will satisfy the needs and constraints of the project. It is also aimed to explain the functional, structural and behavioral features of the system by using specific types of UML diagrams such as class, sequence, state diagrams. In order to support these diagrams, graphical user interface prototypes are also provided in the document.\\
\skipsubsection


\subsection{Intended Audience}
\paragraph{}
\normalsize
This document is intended for both the stakeholders and the developers who build the system.\\
\skipsubsection

\subsection{References}
	\begin{enumerate}
		\item IEEE. IEEE Std 1016-2009 IEEE Standard for Information Technology – System Design – Software Design Descriptions. IEEE Computer Society, 2009.
		\item StarUML 5.0 User Guide. \url{http://staruml.sourceforge.net/docs/user-guide(en)/toc.html}
	\end{enumerate}
\skipsubsection
